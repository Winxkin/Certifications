\chapter{KẾT LUẬN VÀ ĐỊNH HƯỚNG PHÁT TRIỂN CHO CHUYÊN ĐỀ 2}
\section{Kết Luận}
Trong quá trình nghiên cứu, học viên đã tiếp cận và hiểu biết sâu hơn về các mô hình mạng nơ-ron, đặc biệt là mạng nơ-ron tích chập và lý thuyết học sâu. Họ cũng đã tìm hiểu về bài toán phân vùng ảnh và cách áp dụng học sâu vào bài toán này. Dựa trên kiến thức thu thập được, học viên đã phát hiện ra rằng mô hình U-Net là sự lựa chọn tốt nhất cho dự án của mình.

U-Net đã chứng minh được hiệu quả và linh hoạt của mình trong việc phân vùng ảnh. So với SqueezeSegV2 và SegNet, U-Net có những ưu điểm nổi bật. Đặc biệt, U-Net có khả năng xử lý tốt các đối tượng nhỏ và linh hoạt trong việc điều chỉnh và tinh chỉnh cho các nhiệm vụ cụ thể. Mô hình này cũng không đối mặt với các hạn chế như SqueezeSegV2, như khả năng nhận diện các vật thể nhỏ và yêu cầu tài nguyên tính toán cao.

Do đó, việc sử dụng U-Net sẽ giúp học viên giải quyết các yêu cầu của dự án một cách hiệu quả và linh hoạt hơn, đồng thời cũng giảm thiểu được các hạn chế mà các mô hình khác như SqueezeSegV2 và SegNet đang đối mặt.

\section{Hướng phát triển}


Trong tương lai, việc nghiên cứu sâu hơn về việc tối ưu hóa cấu trúc mạng và các hàm mất mát có thể giúp cải thiện hiệu suất của mô hình U-Net. Các nghiên cứu này có thể tập trung vào việc tinh chỉnh kiến trúc mạng để tối ưu hóa việc trích xuất đặc trưng và dự đoán. Ngoài ra, cải tiến các hàm mất mát cũng có thể giúp tăng độ chính xác của mô hình và giảm thiểu thời gian dự đoán.

Thêm vào đó, việc thu thập một tập dữ liệu đa dạng hơn cũng là một phần quan trọng trong quá trình nghiên cứu. Dữ liệu này nên bao gồm các vị trí đối tượng lớn hơn và các trường hợp bổ sung của các lớp thiểu số, để mô hình có khả năng học được từ những trường hợp đa dạng và phong phú hơn.

Khám phá các độ phân giải dữ liệu khác nhau cũng là một hướng nghiên cứu tiềm năng. Việc này có thể giúp phát hiện và giải quyết các nhược điểm trong việc phân đoạn ảnh và cung cấp những hiểu biết có giá trị để cải thiện hiệu suất của mô hình U-Net trong các ứng dụng thực tế.